\documentclass[a4paper,10pt]{article}
\usepackage[utf8]{inputenc}
\usepackage{amsmath}

%opening
\title{}
\author{Arnaud Paran}

\begin{document}

  \maketitle

  \begin{abstract}

  \end{abstract}

  \section{Mémo}
  technique à utiliser gradient conjugué

  \section{Étude de l'équation simplifiée}
      
  \subsection{Équation d'Exner}
  Voici l'équation d'Exner, avec un terme $\underline{q}$ à expliciter
  \begin{equation}\label{eq}
    \phi\frac{\partial z}{\partial t}+div\underline{q}=0
  \end{equation}
  On prend un terme $\underline{q}$ de la forme
  \begin{equation}
    \underline{q}=\underline{u}(|\underline{u}|^{\frac{1}{2}}-|\underline{u}_c|^{\frac{1}{2}})(1-\beta_*\frac{\partial z}{\partial s})
  \end{equation}
  Or, on a 
  \[\partial_sz=\frac{\underline{u}}{|\underline{u}|}.\underline\nabla z\]
  On doit donc résoudre
  \[\phi\frac{\partial z}{\partial t}+div(\underline{u}(|\underline{u}|^{\frac{1}{2}}-|\underline{u}_c|^{\frac{1}{2}})(1-\beta_*\frac{\underline{u}}{|\underline{u}|}.\underline\nabla z))=0\]
  
\end{document}
