\documentclass[a4paper,10pt]{article}
\usepackage[utf8]{inputenc}
\usepackage{amsmath}
\usepackage{mathtools}

%opening
\title{}
\author{Arnaud Paran}

\begin{document}

  \maketitle

  \begin{abstract}

  \end{abstract}

  \section{Mémo}
  technique à utiliser gradient conjugué

  \section{Étude de l'équation simplifiée}
      
  \subsection{Obtention du systèmé d'équations}
  Voici l'équation d'Exner, avec un terme $\underline{q}$ à expliciter
  \begin{equation}\label{eq}
    \phi\frac{\partial z}{\partial t}+div\underline{q}=0
  \end{equation}
  On prend un terme $\underline{q}$ de la forme
  \begin{equation}
    \underline{q}=\underline{u}(|\underline{u}|^{\frac{1}{2}}-|\underline{u}_c|^{\frac{1}{2}})(1-\beta_*\frac{\partial z}{\partial s})
  \end{equation}
  Or, on a 
  \[\partial_sz=\frac{\underline{u}}{|\underline{u}|}.\underline\nabla z\]
  On doit donc résoudre
  \[\phi\frac{\partial z}{\partial t}+div(\underline{u}(|\underline{u}|^{\frac{1}{2}}-|\underline{u}_c|^{\frac{1}{2}})(1-\beta_*\frac{\underline{u}}{|\underline{u}|}.\underline\nabla z))=0\]
  En linéarisant, et en utilisant la formule de la divergence
  \[\phi\frac{\partial z}{\partial t}+div(\underline{u})(1-\beta_*\psi)-\beta_*\underline{u}.\underline{\nabla}\psi=0\]
  On passe au gradient pour obtenir
  
  \begin{equation}
    \left\{\begin{array}{l}
      \phi\frac{\partial \psi}{\partial t}-\beta_*\underline{s}.\underline{\nabla}(div(\underline{u}))\psi-div(\underline{u})\beta_*\underline{s}.\underline\nabla(\psi)-\beta_*\underline{s}.\underline{\nabla}(\underline{u}.\underline{\nabla}\psi)=-\underline{s}.\underline{\nabla}(div(\underline{u}))\\
      \phi\frac{\partial \psi^\perp}{\partial t}-\beta_*\underline{t}.\underline{\nabla}(div(\underline{u}))\psi-div(\underline{u})\beta_*\underline{t}.\underline\nabla(\psi)-\beta_*\underline{t}.\underline{\nabla}(\underline{u}.\underline{\nabla}\psi)=-\underline{t}.\underline{\nabla}(div(\underline{u}))
    \end{array}\right.
  \end{equation}
  
  \subsubsection{Terme de relaxation}
  On a un terme de relaxation
  \[-\beta_*\underline{\nabla}(div(\underline{u}))\psi\]
  On remarque que si $\phi\beta_*\underline s\underline{\nabla}(div(\underline{u}))\geq0$ ou
  $\phi\beta_*\underline s\underline{\nabla}(div(\underline{u}))\geq0$, nos solutions divergent. D'autre part, si on obtient
  une solution non divergente, il faudra aussi contrôler le rapport
  $\Delta t\leq\frac\phi{\beta_*|\underline\nabla(div(\underline u))|}$
  \subsubsection{Terme d'advection}
  On a aussi un terme d'advection
  \[-div(\underline{u})\beta_*\underline\nabla(\psi)\]
  Il faudra le contrôler à l'aide d'un schéma décentré et de la jauge $\Delta t\leq \frac{\phi\Delta x}{div(\underline u) \beta_*}$
  \subsubsection{Gradient de produit scalaire}
  Un terme qu'on connaît peu
  \[-\beta_*\underline{\nabla}(\underline{u}.\underline{\nabla}\psi)\]
  \subsubsection{Terme source}
  Et le terme source
  \[-\underline{\nabla}(div(\underline{u}))\]
\end{document}
