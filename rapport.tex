\documentclass[a4paper,10pt]{article}
\usepackage[utf8]{inputenc}
\usepackage{amsmath}
\usepackage{mathtools}

%opening
\title{}
\author{Arnaud Paran}

\begin{document}
\makeatletter
\newcommand*{\bdiv}{%
  \nonscript\mskip-\medmuskip\mkern5mu%
  \mathbin{\operator@font div}\penalty900\mkern5mu%
  \nonscript\mskip-\medmuskip
}
\makeatother
  

  \maketitle

  \begin{abstract}

  \end{abstract}

  \section{Mémo}
  technique à utiliser gradient conjugué

  \section{Étude de l'équation simplifiée}
      
  \subsection{Obtention du systèmé d'équations}
  Voici l'équation d'Exner, avec un terme $\underline{q}$ à expliciter
  \begin{equation}\label{eq}
    \phi\frac{\partial z}{\partial t}+div\underline{q}=0
  \end{equation}
  On prend un terme $\underline{q}$ de la forme
  \begin{equation}
    \underline{q}=\underline{u}(|\underline{u}|^{\frac{1}{2}}-|\underline{u}_c|^{\frac{1}{2}})(1-\beta_*\frac{\partial z}{\partial s})
  \end{equation}
  Or, on a 
  \[\partial_sz=\frac{\underline{u}}{|\underline{u}|}.\underline\nabla z\]
  On doit donc résoudre
  \[\phi\frac{\partial z}{\partial t}+div(\underline{u}(|\underline{u}|^{\frac{1}{2}}-|\underline{u}_c|^{\frac{1}{2}})(1-\beta_*\frac{\underline{u}}{|\underline{u}|}.\underline\nabla z))=0\]
  En linéarisant, et en utilisant la formule de la divergence
  \[\phi\frac{\partial z}{\partial t}+div(\underline{u})(1-\beta_*\psi)-\beta_*\underline{u}.\underline{\nabla}\psi=0\]
  On passe au gradient pour obtenir
  
  \begin{equation}
    \left\{\begin{array}{l}
      \phi\frac{\partial \psi}{\partial t}-\beta_*\underline{s}.\underline{\nabla}(div(\underline{u}))\psi-div(\underline{u})\beta_*\underline{s}.\underline\nabla(\psi)-\beta_*\underline{s}.\underline{\nabla}(\underline{u}.\underline{\nabla}\psi)=-\underline{s}.\underline{\nabla}(div(\underline{u}))\\
      \phi\frac{\partial \psi^\perp}{\partial t}-\beta_*\underline{t}.\underline{\nabla}(div(\underline{u}))\psi-div(\underline{u})\beta_*\underline{t}.\underline\nabla(\psi)-\beta_*\underline{t}.\underline{\nabla}(\underline{u}.\underline{\nabla}\psi)=-\underline{t}.\underline{\nabla}(div(\underline{u}))
    \end{array}\right.
  \end{equation}
  
  \subsubsection{Terme de relaxation}
  On a un terme de relaxation
  \[-\beta_*\underline{\nabla}(div(\underline{u}))\psi\]
  On remarque que si $\phi\beta_*\underline s\underline{\nabla}(div(\underline{u}))\geq0$ ou
  $\phi\beta_*\underline s\underline{\nabla}(div(\underline{u}))\geq0$, nos solutions divergent. D'autre part, si on obtient
  une solution non divergente, il faudra aussi contrôler le rapport
  $\Delta t\leq\frac\phi{\beta_*|\underline\nabla(div(\underline u))|}$							
  \subsubsection{Terme d'advection}
  On a aussi un terme d'advection
  \[-div(\underline{u})\beta_*\underline\nabla(\psi)\]
  Il faudra le contrôler à l'aide d'un schéma décentré et de la jauge $\Delta t\leq \frac{\phi\Delta x}{div(\underline u) \beta_*}$
  \subsubsection{Gradient de produit scalaire}
  Un terme qu'on connaît peu
  \[-\beta_*\underline{\nabla}(\underline{u}.\underline{\nabla}\psi)\]
  \subsubsection{Terme source}
  Et le terme source
  \[-\underline{\nabla}(div(\underline{u}))\]
  \section{résultats du 28 avril}
  Je me suis aujourd'hui intéressé au terme $-\beta_*\underline{\nabla}(\underline{u}.\underline{\nabla}\psi)$
  J'ai montré que pour deux champs eulériens,
  \[\underline\nabla(\underline a.\underline b)=^t\underline{\underline\nabla}\underline a.\underline b+^t\underline{\underline\nabla}\underline b.\underline a\]
  On se retrouve donc avec
  \[-\beta_*^t\underline{\underline\nabla}\underline u.\underline\nabla \psi-\beta_*^t\underline{\underline\nabla}(\underline\nabla \psi).\underline u\]
  Or, par symétrie de la Hessienne, on a
  \[-\beta_*^t\underline{\underline\nabla}\underline u.\underline\nabla \psi-\beta_*\underline{\underline\nabla}(\underline\nabla \psi).\underline u\]
  \section{Le rapport}
  \begin{abstract}
   Ce travail consiste en l'étude numérique du transport de sédiments dans un canal rectiligne. Nous avons utilisé un cas test et nous
   l'avons post traité afin de vérifier les équations de saint venant et d'Exner. Les scripts ont été codés en python à partir de
   données issue de Telemac sur le cas de Lanzoni.
  \end{abstract}
  \section{Notations}
  $x$ : abscisse dans le canal
  $y$ : ordonnée dans le canal
  $z$ : hauteur du fond du canal
  $h$ : hauteur d'eau
  $u$ : vitesse selon x de l'eau
  $v$ : vitesse selon y de l'eau
  $g$ : accélération de la pesanteur
  $K$ : 
  $\underline u$ : la vitesse vectorielle de l'eau
  \section{Introduction}
  Le transport de sédiments est un domaine de grande importance. En effet, prévoir correctement le niveau du fond des rivières permet
  d'optimiser les opérations code dragages qui sont très coûteuses.
  
  \section{Modèle}
  Nous modélisons le fleuve comme un canal rectiligne de longueur 120m et de largeur 1,5m
  
  Ce canal suit les équations de Saint Venant, c'est à dire:
  \begin{equation}
    \frac{\partial \underline u}{\partial t}+\bdiv(h\underline u)
  \end{equation}
  \begin{equation}
    \frac{\partial h\underline u}{\partial t}+\bdiv(h\underline u \otimes \underline u)+\nabla(\frac 1 2 gh^2)+gh(\underline S_0 + \underline S_f)=0
  \end{equation}
  
  Ici, $\underline S_0$ est la pente du lit de la rivière et $\underline S_f$ le frottement.
  la pente s'exprime par
  \begin{equation}
   \underline S_0 = \bdiv z
  \end{equation}
  \begin{equation}
    \underline S_{fx}=\frac{u^2+v^2}{K^2h^{\frac 4 3}}u
  \end{equation}
  \begin{equation}
    \underline S_{fy}=\frac{u^2+v^2}{K^2h^{\frac 4 3}}v
  \end{equation}
  \section{Vérification de la stationnarité}
  Nous avons tenté de vérifier que les résultats issus de Télémac vérifiaient les équations de Saint Venant.
  Pour ce faire, nous avons chargé les données dans un tableau python. Ensuite, nous avons calculé les gradients des fonctions
  formes sur chaque simplex avec de formules simples obtenues géométriquement en considérant la hauteur du simplex.
  Les indices représentent le point du simplex
  \begin{equation}
    g_1=\frac{(y_3-y_2) \underline e_x + (x_2-x_3) \underline e_y}{(x_2-x_3)*(y_1-y_3)-(x_1-x_3)*(y_2-y_3)}
  \end{equation}
  \begin{equation}
    g_2=\frac{(y_3-y_1) \underline e_x + (x_1-x_3) \underline e_y}{(x_1-x_3)*(y_2-y_3)-(x_2-x_3)*(y_1-y_3)}
  \end{equation}
  \begin{equation}
    g_3=\frac{(y_1-y_2) \underline e_x + (x_2-x_1) \underline e_y}{(x_2-x_1)*(y_3-y_1)-(x_3-x_1)*(y_2-y_1)}
  \end{equation}
  
  Cela nous a permis donc de considérer la dérivation des grandeurs physiques du problème.
  
  Sur les données de départ, à partir d'environ 60 mètres en partant de l'amont, la vitesse de l'eau dans le canal
  semble ``stationnaire''. Nous avons donc décidé de nous restreindre à cette partie du canal afin d'annuler les dérivées
  temporelles dans nos calculs. Nous n'aurions pas pu considérer de telles dérivées car les données analysées étaient fournies à
  un instant donné, donc sans dépendance au temps.
  La restriction à une partie du canal a été simplement faite en ne chargeant en mémoire que les simplex se trouvant à plus de
  60 mètres de l'amont. La matrice de connectivité représentant les simplex a du être changée aussi car n'ayant chargé que les
  points à plus de 60 mètres, il a fallu changer l'indice des lignes.
  
  Nous nous sommes alors attaqués à la première équation, la conservation de la matière.
  Elle s'est réécrite
  \begin{equation}
    \underline u . \underline \nabla h + h\bdiv(\underline u) = 0
  \end{equation}
  Nous avons donc calculé les deux termes et les avons additionnés. Ensuite, nous les avons représentés graphiquement.
  
  Or, nous nous intéressons ici à des points relatifs à un simplex, on obtient donc plusieurs valeurs pour un point donné.
  Il y a donc discontinuité du tracé qui devient difficile à lire. Afin de résoudre ce problème, nous avons moyenné sur
  chaque point les différentes valeurs par simplex obtenues afin d'obtenir une unique valeur de la somme en chaque point.
  
  Nous avons trouvé des valeurs de l'ordre de 1000, ce qui est loin du 0 auquel nous nous attendions, cependant, ce que l'on voulait
  c'était
  \begin{equation}
    \underline u . \underline \nabla h + h\bdiv(\underline u) \leq \frac{\varepsilon}{\Delta t}
  \end{equation}
  Or, le $\Delta t$ pris pour l'expérimentation était de l'ordre de $10^{-3}$.
  
\end{document}
